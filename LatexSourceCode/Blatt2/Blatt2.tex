\documentclass[a4paper,12pt]{article}
\usepackage{minted}
\usemintedstyle{vs}
\usepackage{fancyhdr}
\usepackage{fancyheadings}
\usepackage[ngerman,german]{babel}
\usepackage{german}
\usepackage[utf8]{inputenc}
%\usepackage[latin1]{inputenc}
\usepackage[active]{srcltx}
\usepackage{algorithm}
\usepackage[noend]{algorithmic}
\usepackage{amsmath}
\usepackage{amssymb}
\usepackage{amsthm}
\usepackage{bbm}
\usepackage{enumerate}
\usepackage{graphicx}
\usepackage{ifthen}
\usepackage{listings}
\usepackage{struktex}
\usepackage{hyperref}
\usepackage{tikz}
\usetikzlibrary{positioning,automata}

%%%%%%%%%%%%%%%%%%%%%%%%%%%%%%%%%%%%%%%%%%%%%%%%%%%%%%
%%%%%%%%%%%%%% EDIT THIS PART %%%%%%%%%%%%%%%%%%%%%%%%
%%%%%%%%%%%%%%%%%%%%%%%%%%%%%%%%%%%%%%%%%%%%%%%%%%%%%%
\newcommand{\Fach}{C\#}
\newcommand{\Name}{Fachschaft}
\newcommand{\Tutorium}{}
\newcommand{\Semester}{OHM }
\newcommand{\KlausurLoesung }{SoSe2020} %  <-- UPDATE ME
%%%%%%%%%%%%%%%%%%%%%%%%%%%%%%%%%%%%%%%%%%%%%%%%%%%%%%
%%%%%%%%%%%%%%%%%%%%%%%%%%%%%%%%%%%%%%%%%%%%%%%%%%%%%%

\setlength{\parindent}{0em}
\topmargin -1.0cm
\oddsidemargin 0cm
\evensidemargin 0cm
\setlength{\textheight}{9.2in}
\setlength{\textwidth}{6.0in}

%%%%%%%%%%%%%%%
%% Aufgaben-COMMAND
\newcommand{\Aufgabe}[1]{
  {
  \vspace*{0.5cm}
  \textsf{\textbf{Aufgabe #1}}
  \vspace*{0.2cm}
  
  }
}

\newcommand{\Definition}[1]{
  {
  \vspace*{0.5cm}
  \textsf{\textbf{#1}}
  \vspace*{0.2cm}
  
  }
}

%%%%%%%%%%%%%%
\hypersetup{
    pdftitle={\Fach{}: Übungsblatt \KlausurLoesung{}},
    pdfauthor={\Name},
    pdfborder={0 0 0}
}

\lstset{ %
language=java,
basicstyle=\footnotesize\tt,
showtabs=false,
tabsize=2,
captionpos=b,
breaklines=true,
extendedchars=true,
showstringspaces=false,
flexiblecolumns=true,
}

\title{Übungsblatt Deterministische Endliche Automaten}
\author{\Name{}}

\begin{document}
\thispagestyle{fancy}
\lhead{\sf \large \Fach{} \\ \small \Name{} }
\rhead{\sf \Semester{} \\  Brückenkurs \Tutorium{}}
\vspace*{0.2cm}
\begin{center}
\LARGE \sf \textbf{Blatt 2}
\end{center}
\vspace*{0.2cm}

%%%%%%%%%%%%%%%%%%%%%%%%%%%%%%%%%%%%%%%%%%%%%%%%%%%%%%
%% Insert your solutions here %%%%%%%%%%%%%%%%%%%%%%%%
%%%%%%%%%%%%%%%%%%%%%%%%%%%%%%%%%%%%%%%%%%%%%%%%%%%%%%
\section{Boolean Variablen}
\Definition{Beispiel Programm}

\begin{minted}{csharp}
bool falsch = false;
bool wahr = true;

Console.WriteLine(falsch);

Console.WriteLine(wahr);

Console.WriteLine(wahr && falsch);

Console.WriteLine(wahr || falsch);

Console.WriteLine(!wahr);
\end{minted}



\Aufgabe{1} 
\begin{enumerate}[a)]
\item
Kopiere die Programmzeilen aus dem Beispiel in Visual Studio und führe sie aus!
\item
Versuche durch ausprobieren mit verschiedenen Belegungen herauszufinden, was der $ || $, der $!$ und was der \&\&  Operator bewirken.
\end{enumerate}

\section{If Else Statements}
\Definition{Beispiel Programm}

\begin{minted}{csharp}
bool isRaining = false;

if (isRaining)
{
    Console.WriteLine("Nimm einen Regenschirm mit.");
}
else 
{
    Console.WriteLine("Einen schönen sonnigen Tag.");
}
\end{minted}

\Aufgabe{2} 
\begin{enumerate}[a)]
\item
Kopiere die Programmzeilen aus dem Beispiel in Visual Studio und führe sie aus!
\item 
Ändere das Programm so ab, dass ''Nimm einen Regenschirm mit''  ausgegeben wird.
\item
Definiere eine weitere Boolean Variable isFreezing. \\
Wenn es kalt ist und regnet gebe aus: ''Ziehe eine Jacke an und nimm einen Schirm mit''. \\
Wenn es kalt ist und  nicht regnet gebe aus: ''Zieh einen Pullover an.'' \\
Wenn es warm ist und regnet gebe aus: ''Singing in the Rain''. 
\end{enumerate}

\section{Vergleichsoperationen}
\Definition{Beispiel Programm}

\begin{minted}{csharp}
bool kleiner = 3 < 4;
Console.WriteLine(kleiner);

bool größer = 25 > 13;
Console.WriteLine(größer);

bool gleich = (3 * 2 + 14) == 22;
Console.WriteLine(gleich);

bool größergleich = 15 >= 17;
Console.WriteLine(größergleich);

bool gleich2 = "Apfel" == "Birne";
Console.WriteLine(gleich2);
\end{minted}

\Aufgabe{3} 
\begin{enumerate}[a)]
\item
Kopiere die Programmzeilen aus dem Beispiel in Visual Studio und führe sie aus!
\item 
Ändere das Programm so ab, dass in jedem Console.WriteLine() true ausgegeben wird.
\item
Ändere das Programm in Aufgabe 2 c) so ab, dass der Nutzer am Anfang gefragt wird ob es regnet oder kalt ist.
\item
Speicher ein selbstausgedachtes Passwort in einer String Variable.
Das Programm fragt nach einem Password beim Nutzer.
Wenn das eingegebene Passwort gleich dem gespeicherten Passwort ist, dann soll dem Nutzer ''access granted'', ansonsten ''access denied'' ausgeben werden.
\item 
Schreibe ein Programm, dass den Nutzer drei Integer Zahlen eingeben lässt. Finde durch Vergleichsoperationen und If-Else Statements heraus, welche den kleinsten Wert hat und gebe diese in die Konsole aus.
\item
Frage den Nutzer welche zwei Zahlen er eingeben möchte. \\
Frage den Nutzer ob er Multiplizieren, Addieren oder Subtrahieren will. \\
Benutze if und else Statements um dem Nutzer das richtige Ergebnis auszugeben. \\

\end{enumerate}

\section{Teilbarkeit}
\Definition{Beispiel Programm}

\begin{minted}{csharp}
int input = 12;

if(input%2==0)
{
    Console.WriteLine("Die Zahl ist durch 2 Teilbar");
}
\end{minted}

\Aufgabe{4} 
\begin{enumerate}[a)]
\item
Kopiere die Programmzeilen aus dem Beispiel in Visual Studio und führe sie aus!
\item 
Ändere das Programm so ab, dass die if-Bedingung nicht erfüllt ist.
\item
Ändere das Programm so ab, dass der Nutzer eine Eingabe macht und für diese Überprüft wird, ob die Zahl durch zwei teilbar ist.
\item
Ändere das Programm so ab, dass auch überprüft wird, ob die Eingabe durch drei und durch fünf teilbar ist. \\
Wenn ja soll ebenfalls ausgegeben werden: ''Die Zahl ist durch 5 teilbar''. \\
Beispiel für die Zahl 15. Ausgabe: \\
''Die Zahl ist durch 3 teilbar'' \\
''Die Zahl ist durch 5 teilbar'' \\
\item
Deklariere eine Boolean Variable isTeilbar mit false. In jedem If Statement wird die Variable auf true gesetzt. \\
Wenn die Variable bis zum Ende auf false bleibt, schreibe: ''Die Eingegebene Zahl ist eine Primzahl'' 
in die Konsole.
\item
Was müsste man tun um das auch für Zahlen größer als 25 zu überprüfen? \\
Wie müsste ein Programm aussehen, dass für beliebig große Zahlen überprüft ob sie eine Primzahl sind?
\end{enumerate}

\section{Double Variablen}
\Definition{Beispiel Programm}

\begin{minted}{csharp}
double zahl = 13.2;

Console.WriteLine(zahl / 3);
\end{minted}

\Aufgabe{5} 
\begin{enumerate}[a)]
\item
Kopiere die Programmzeilen aus dem Beispiel in Visual Studio und führe sie aus!
\item
Probieren mit der Double Variablen alle Rechenoperationen aus.
\item
Versuche mit Google eigenständig herauszufinden, wie man eine String Variable in eine Double Variable Konvertieren kann.
\item
Frage den Nutzer welche zwei Zahlen er eingeben möchte. \\
Frage den Nutzer welche Rechenoperation er ausführen möchte. \\
Benutze if und else Statements um dem Nutzer das richtige Ergebnis auszugeben. \\
Wenn der Nutzer durch 0 teilen möchte, gib stattdessen einen Fehlertext in der Konsole aus.

\end{enumerate}

\section{Rechenoperationen}
\Definition{Beispiel Programm}

\begin{minted}{csharp}
Console.WriteLine(Math.Pow(3,4));
\end{minted}

\Aufgabe{6} 
\begin{enumerate}[a)]
\item
Kopiere die Programmzeile aus dem Beispiel in Visual Studio und führe sie aus! Versuche herauszufinden, was die Math.Pow() Funktion bewirkt.
\item
Versuche mit Google eigenständig herauszufinden, wie man die Fakultätsfunktion berechnen kann.
\item
Versuche mit Google eigenständig herauszufinden, wie man die Wurzelfunktion berechnen kann.
\item
Verändere das Programm aus 5 d) so, dass auch die Wurzelfunktion und die Potenzfunktion berechnet werden kann.
\end{enumerate}

%%%%%%%%%%%%%%%%%%%%%%%%%%%%%%%%%%%%%%%%%%%%%%%%%%%%%%
%%%%%%%%%%%%%%%%%%%%%%%%%%%%%%%%%%%%%%%%%%%%%%%%%%%%%%
\end{document}

